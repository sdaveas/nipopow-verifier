\chapter{Introduction}

\section{Motivation}

Digital coins are peer-to-peer currencies based on applied cryptography
for the validation of transactions. Most of them are based on
blockchain(ref), a form of decentralized database. In this database, a
public ledger is deployed which is stored and updated by thousands of
users in absence of supervision from public authorities.

In 2008, Bitcoin(ref), the first ever successful decentralized digital
coin, was invented by an unknown person or group of people using the
name Satoshi Nakamoto. A year after, the bitcoin network started,
quickly followed by several other digital coins, which in the
cryptocurrency folklore are known as altcoins. Usually, altcoins are
based on innovative features previously missing from the cryptocurrency
market, and they are either accepted or rejected by the community.
Popular altcoins are Ethereum(ref), which is the first to provide smart
contracts, Ripple(ref) which provides real-time payment settlements and
Litecoin(ref) which enables near-zero cost payments.

Over the last decade, cryptocurrencies gained attention from the public
as an increased number of users accept and trust decentralized
transactions. Specifically, in 2017, the popularity of cryptocurrencies
rapidly grew, resulting in massive capitalisation and creation of
tokens. During this period, some of the issues that blockchain
technology faces were displayed. One of these issues is blockchain
interoperability, the property of distinct blockchains to interact
efficiently with each other. Despite its great importance, this field
has not been addressed until recently. To date, cryptocurrencies are
lacking a commonly accepted protocol that enables distributed
interoperability. Such a protocol would be very useful to blockchain
technology, since it would allow users to variously utilize features of
different blockchains. For example, one can store their funds in
Bitcoins, and convert them to Ether to make a payment, benefiting from
lower transaction fees and quicker transaction rates.

A crosschain protocol would enable two main operations
\begin{itemize}
    \item
        Crosschain trading: An entity with deposits in blockchain A, makes a
        payment to an entity at blockchain B.
    \item
        Crosschain fund transfer: Entity transfers owning funds from
        blockchain A to blockchain B. After this operation, the funds no
        longer exist at blockchain A. The entity can return any portion of the
        original amount to the blockchain of origin.
\end{itemize}

Currently, this operation is only available to the users via third party
applications, such as multi-currency wallets. This treatment certainly
opposes to the nature of blockchain, which is a decentralized construction.
This motivated us to create a solution that enables cheap, trust-less
crosschain operations.

\section{Previous Work}

In this paper, we focus on recent research in the area of crosschains.
In particular, we make use of the cryptographic primitive
Non-Interactive Proofs of Proof of Work (NIPoPoWs)(ref), which enables
the compression of a chain to its poly-logarothmic size. NIPoPoWs is the
main building block of our solution in order to make the occurrence of
an event of blockchain A provably known to blockchain B.

We are based on previous work done by Giorgos Christoglou et al.(ref),
which was the first ever implementation of crosschain events
verification. The work of Giorgos et al.\ focuses on verifying Bitcoin
events from the Ethereum blockchain. In order to provide this
functionality, a NIPoPoW verifier was developed in Solidity(ref), one of
the programming languages of Ethereum blockchain. This solution,
however, is impossible to be applied in a real blockchain due to
extensive gas usage and severe security issues.

\todo{maybe "severe" is too harsh}

\section{Our contributions}

A series of keen observations, the application of gas-efficient
practices and the utilization of modern Solidity features led us to the
design of a new verifier architecture. This allowed us to repair
previous vulnerabilities and enable crosschain operations by providing a
secure, superlight Bitcoin client that can be deployed on the real
blockchain.

Our contribution is the creation of a Bitcoin client in Solidity that
verifies events across blockchains while meeting the following criteria:
\begin{itemize}
    \item
        Security: Is secure against any adversarial attack.
    \item
        Trustless: Does not have dependencies at any third-party applications.
    \item
        Applicability: Is applicable to the real blockchain. That is, data
        derived from the full-sized Bitcoin blockchain are successfully
        accepted and processed by the client without exceeding the build-in
        constraints of the Ethereum blockchain (i.e.\ block gas limit, calldata
        limit etc).
    \item
        Cheap: Is cheaper than the current state of the art technologies. This
        would make trustless crosschain transactions more popular and
        affordable.
\end{itemize}

\pagebreak
