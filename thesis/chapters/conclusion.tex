\chapter{CONCLUSION} \SECTION{SUMMARY}

In this work we present a gas-efficient Bitcoin superlight client in Solidity.
The core of our construction is the cryptographic primitive Non-Interactive
Proofs of Proof of Work which is based on the notion of superblocks. Our work
is the first practical implementation of NIPoPoW verification in Solidity; by
constructing such a tool, we deem the NIPoPoW protocol to be practical. In the
process of implementing our client, we developed several techniques that
leveraged the decrease of gas consumption. All of our proposals address the
gas-efficient implementation of the protocol in Solidity. We claim that one of
these paradigms, a design pattern we term \emph{hash-and-resubmit}, may also be
applicable to other smart contracts in order to facilitate gas saving and lead
to better performance. The rest of the techniques are developed are explicit
refinements to the NIPoPoW protocol. Overall, we have have achieved to
eliminate the vast majority of storage variables of the contract, and to make
the complexity of each phase (submit and contest) constant to the size of the
proof by leveraging an optimistic scheme. These refinements were the key to
deliver a gas-efficient implementation, and are the main difference in
comparison with the previous work. Finally, we present a cryptographic analysis
of our work in several scenarios, the result of which deem our
implementation to be practical.

\section{Discussion and Future Work}

Towards the completion of your primary goal, which has been the topic of this
thesis, several new questions surfaced; we list those ideas below as relevant
future work regarding the implementation of practical superlight clients, as
the next steps towards the establishment of the blockchain interoperability.

\subsection{Implementation Under a Velvet Fork}

In our work, the interlink structure is stored in the block header of each
block. Hence, we are addressing a hard fork. These kind of fork are improbable
to be adopted by the Bitcoin community. However, a soft fork or a velvet fork
can also be used to encapsulate the information needed in a manner that does
not affect the size of block headers; this can be done by utilizing the extra
bytes in the \emph{coinbase} transaction. In fact, recent
work~\cite{velvet-nipopows} have shown that by adding the root of a Merlke
Mountain Range in the information of each block is sufficient to ensure
structural correctness of the proof under a velvet fork.

Naturally, the process of this information by the verifier will result to some
gas increase, since the verification of the MMR will also be needed in order to
ensure that submitted proofs are valid. We claim that, since the verification
of proofs is now constant, this cost will be negligible, however we do not
illustrate this in our implementation.

\subsection{Proceeding to Bitcoin Transaction Verification}

In this thesis, we focus on verifying the existence of blocks inside a chain.
On can claim that this contribution is not sufficient for a realistic
application since users are mostly concerned about the verification of more
explicit information, such as the existence of a transaction.

Similarly to the implementation of a velvet fork, we claim that the addition of
a transaction verification will not add significant burden in terms of gas,
since this operation is constant to the size of the blockchain (and, hence, to
the size of the proof). In bitcoin, the verification of a transactions is
equivalent to the verification of a Merkle Tree. We have observed that the gas
cost for a simple Merkle Tree inclusion proof is negligible; however, the scope
of this thesis does not include the implementation of transaction verification.
We strongly believe that the addition of this feature will not change the state
of the gas consumption of out client significantly, but we leave the
experimental demonstration for future work.

\subsection{Verify Ethereum Events}

A logical step after the implementation of a practical Bitcoin superlight
client, is the implementation of an Ethereum superlight client. We understand
that the verification of NIPoPoWs for Ethereum events will be significantly
more challenging than the verification of Bitcoin NIPoPoWs due to the
considerably larger size of Ethereum block headers size. Most probably, new
techniques will to be adopted, such as shading and multi-stage submission.  We
believe that the optimizations we introduce in the scope of implementing a
Bitcoin superlight client will be useful for the Ethereum superlight client. We
leave the implementation of Ethereum NIPoPoW verification as a future work.

\subsection{Formally Prove Protocol Changes}

Towards creating our client, we have proceeded to several protocol refinements.
Although we strongly believe that the security of the protocol is not affected,
we do not provide formal security proofs for this claim. We leave the
construction of formal proofs that prove the security under our refined
protocol as a future work.

\subsection{Hash-and-Resubmit Variations}

As we already discussed, there is a variation to the \emph{hash-and-resubmit}
pattern that uses Merkle Trees. Although this variation does not beat the plain
version of the pattern in terms of performance, we believe that different
policies regarding the segmentation of the dispatched data can lead to better
performance. Specifically, we believe that different chunk size will leverage
less demanding hashing operations. We also leave this research as a future
work.
