\section{Leveraging An Optimistic Schema}

We discussed how the verification in the NIPoPoW protocol is realized in two
phases. In \emph{submit} phase, the verification of the $\pis$ is performed.
This is necessary in order to prevent adversaries from injecting blocks that do
not belong to the chain, or changing existing blocks. A proof is valid for
submission if it is \emph{structurally correct}. The correctly structured
NIPoPoW has the following requirements:

\begin{enumerate}
    \item The first block of the proof is the genesis block of the underlying
        blockchain.
    \item Every block has a valid interlink.
\end{enumerate}

Asserting the existence of genesis in the first index of a chain is an
inexpensive operation of constant complexity. However, confirming the interlink
correctness of all blocks is a process of linear complexity to the size of the
proof. Albeit the verification is performed in memory, sufficiently large
proofs result into costly submissions, and it consists the most demanding
function of \emph{submit} phase. In Table~\ref{tab:valid-interlink-cost} we
display the cost of \textsf{valid-interlink} function which determines the
structural correctness of a proof in comparison with the overall gas used in
\textsf{submit}.

\begin{table}[h]
\begin{tabular}{|c|c|c|}
\hline
\textbf{Process} & \textbf{Gas cost} & \multicolumn{1}{l|}{\textbf{Total \%}} \\ \hline
\textsf{verify-interlink} & 2,200,000         & 53\%                                     \\ \hline
\textsf{submit}           & 4,700,000         & 100\%                                    \\ \hline
\end{tabular}
\caption{Gas usage of function \textsf{verify-interlink} compared to overall
gas consumption of \textsf{submit}.}
\label{tab:valid-interlink-cost}
\end{table}


\newcommand{\dispute}{\emph{dispute\ }} \noindent \textbf{Dispute phase.} We
observe that the addition of a phase in our protocol alleviates the burden of
verifying all elements of the proof by enabling the indication of an incorrect
block. This phase, which we term \dispute phase leverages selective
verification of the submitted proof at a certain index, which, as a constant
operation, significantly reduces the gas cost of the verification process.

In the protocol where \emph{dispute} is incorporated, when an invalid proof
$\pis$ is submitted by $\es$, a node, $\ec$, retrieves the proof from the
calldata. Then, the proof is checked for its validity \emph{off-chain}. In order
to prove that $\pis$ is invalid, $\ec$ only needs to indicate the index in
which $\pis$ fails the interlink verification. In turn, $\ec$ calls
$\textsf{dispute}$($\pisa$, $i$), where $i$ indicates the disputing index of
$\pisa$. Therefore, only one block is interpreted \emph{on-chain}, contrary to
the entire span of $\pisa$.

Note that, this additional phase does not imply increased rounds of
interactions between $\es$ and $\ec$. In the case where $\pis$ is invalidated
by \emph{dispute} phase, \emph{contest} phase is skipped.  Similarly, in the
case in which $\pis$ is structurally correct, but represents a chain that is
not honest, then $\ec$ proceeds directly to \emph{contest} phase.

In Table~\ref{tab:dispute-cost} we display the gas consumption for
two independent cycles of interactions:
\begin{enumerate}
    \item Phases \emph{submit} + \emph{dispute} for a case where $\pis$
is structurally incorrect,
    \item Phases \emph{submit} + \emph{contest} for a case where
$\pis$ is structurally correct, but represents a dishonest chain.
\end{enumerate}
\noindent
In Algorithm~\ref{alg:dispute-best-level}, we show the implementation of
\emph{dispute} phase in \textsf{submit} and \textsf{valid-single-interlink}. In
Figure ~\ref{fig:dispute-best-level}, we illustrate the performance gain of the
client using \emph{dispute} phase. The \textsf{contest} function remains
unchanged.

\begin{table}[]
\begin{tabular}{|c|c|}
\hline
\textbf{Function} & \textbf{Gas cost}              \\ \hline
\textsf{submit}            & 2,500,000                      \\ \hline
\textsf{dispute}           & \multicolumn{1}{l|}{1,300,000} \\ \hline
\textbf{Total}             & 3,800,000                      \\ \hline
\end{tabular}
\caption{Gas usage after removing linear block validation using dispute phase.}
\label{tab:dispute-cost}
\end{table}


\begin{algorithm}
    \caption{\label{alg:dispute-best-level}The \textsf{NIPoPoW} client enhanced
        with dispute phase and best-level contesting}

    \begin{algorithmic}[1]

    \Contract{crosschain}
    \State ...
    \Function{\sf submit}{$\pis$, $e$}
        \State \textsf{require}($\pis$[0] = $\genesis$)
        \State \textsf{require}($\textsf{events$[e]$} = \bot$)
        \State \textsf{require}(\textsf{evaluate-predicate}(\textsf{$\pis$}, $e$))
        \State \textsf{events$[e]$.hash} $\gets$ \textsf{H}($\pis$)
    \EndFunction
    \Function{\sf dispute}{$\pisa$, $e$, $i$}\label{alg:dispute-best-level:dispute-start}
        \Comment{$i$: Dispute index}
        \State \textsf{require}(\textsf{events}$[e]$ $\ne$ $\bot$)
        \State \textsf{require}(\textsf{events$[e]$.hash} $=$ \textsf{H}($\pisa$))
        \State \textsf{require}($\neg \textsf{valid-single-interlink}(\pis, i)$)
        \State \textsf{events$[e]$} $\gets$ $\bot$
    \EndFunction\label{alg:dispute-best-level:dispute-end}
    \Function{\sf valid-single-interlink}{$\pi$, $i$}
        \State $\mu\gets\textsf{level}(\pi[i])$
        \State\Return{$\pi[i{+}1].\mathsf{intelink}[\mu] = \pi[i]$}
    \EndFunction
    \Function{\sf contest}{$\pisa$, $\pitrl$, $e$, $f$}
        \State \textsf{require}(\textsf{events}$[e]$ $\ne$ $\bot$)
        \State \textsf{require}(\textsf{events$[e]$.hash} $=$ \textsf{H}($\pisa$))
        \State \textsf{require}(\textsf{valid-interlinks}($\pitrl$))
        \State \textsf{require}(\textsf{minimal-fork}($\pisa$,
        $\pitrl$, $f$))
        \State \textsf{require}(\textsf{arg-at-level}($\pitrl$)
        $>$ \textsf{best-arg}($\pisa[f{:}]$))
        \State \textsf{require}(\textsf{$\neg$evaluate-predicate}($\pitrl$, $e$))
        \State \textsf{events$[e]$} $\gets$ $\bot$
    \EndFunction
    \Function{\sf arg-at-level}{$\pi$}
        \State $\mu \gets \textsf{level}(\pi[-1])$
        \Comment{Pick proof level from a block}
        \For{$b \in \pi$}
            \State$\textsf{assert}(\textsf{level}(b) = \mu)$
        \EndFor
        \State \Return{$2^\mu |\pi|$}
    \EndFunction
    \EndContract
    \vskip8pt
    \end{algorithmic}
\end{algorithm}



\begin{figure}[!h]
    \begin{center}
        \includegraphics[width=1\columnwidth]{figures/blocks-of-best-level.pdf}
    \end{center}
    \caption{Fork of two proofs. Colored blocks after the lca determine the
    score of each proof. Black blocks belong to the level that
    has the best score. Only such blocks are part of the contesting proof.}
    \label{fig:score-at-levels}
\end{figure}

\begin{figure}[!h]
    \begin{center}
        \includegraphics[width=1\columnwidth]{figures/dispute-best-level.pdf}
    \end{center}
    \caption{Caption}
    \label{fig:dispute-best-level}
\end{figure}

