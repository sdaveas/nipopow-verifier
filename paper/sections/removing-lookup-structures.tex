\section{Removing Look-up Structures}

Now that we freely eliminate large arrays, we can focus on other
types of storage variables. The challenge we face is that the protocol of
NIPoPoWs depends on a Directed Acyclic Graph (DAG) of blocks which is a
mutable hashmap. This DAG is needed because interlinks of superblocks can be
adversarially defined. By using DAG, the set of ancestor blocks of a block is
extracted by performing a simple graph search. For the evaluation of the
predicate, the set of \emph{ancestors} of the best blockchain tip is used.
Ancestors are created to avoid an adversary who presents an honest chain but
skips the blocks of interest.

This logic is intuitive and efficient to implement in most traditional
programming languages such as C++, JAVA, Python, JavaScript, etc. However, as
our analysis demonstrates, such an implementation in Solidity is significantly
expensive. Albeit Solidity supports constant-time look-up structures, hashmaps
are only contained in storage. This affects the performance of the client,
especially for large proofs.

We make a keen observation regarding potential positions of the \emph{block of
interest} in proofs, which leads us to the construction of an architecture that
does not require a DAG, the ancestors or other complementary structures. To
support this claim, we adopt the notation from~\cite{nipopows}. We also
consider the predicate $\pred$ to be of the type: ``does block $\boi$ exist
inside proof $\pr$?'', where $\boi$ denotes the block of interest of proof
$\pr$. The entity that performs the submission is $\es$, and the entity that
initiates a contest is $\ec$.

\noindent \textbf{Position of block of interest.} NIPoPoWs are sets of sampled
interlinked blocks, meaning that they can be perceived as chains. Since proofs
$\pis$ and $\pic$ differ, a fork is created at the index of
their last common ancestor ($\lca$). The block of interest lies at a certain
index within $\pis$ and indicates a stable predicate~\cite{nipopows,
generic-client} that is $\true$ for $\pis$. A submission in which $\boi$ is
absent from $\pis$ is aimless, because it automatically fails since no element
of $\pis$ satisfies $\pred$. On the contrary, $\pic$ tries to prove the
\emph{falseness} of the underlying predicate. This means that, if the block of
interest is included in $\pic$, then the contest is aimless. We freely use the
term aimless to also characterize components that are included in such actions
i.e.\ aimless proof, aimless blocks etc. We use the term meaningful to describe
non-aimless actions and components.

\newcommand{\block}{\mathsf{B}}

In the NIPoPoW protocol, proofs' segments $\pis\{{{:}}\lca\}$ and
$\pic\{{:}\lca\}$ are merged to prevent adversaries from skipping
blocks, and the predicate is evaluated against $\pis\{{:}\lca\} \cup
\pic\{{:}\lca\}$. We observe that $\pic\{{:}\lca\}$ can be omitted, because no
block $\block$ exists such that \{$\block : \block \notin \pis\{{:}\lca\} \land
\block \in \pic\{{:}\lca\}$\} where $\block$ results into positive evaluation
of the predicate. This is due to the fact that, in a meaningful contest, $\boi$
is not included in $\pic$. Consequently, $\pic$ is only meaningful if it forks
$\pis$ at a block that is prior to $\boi$.

\renewcommand{\block}{}

\begin{figure}[h]
    \begin{center}
        \includegraphics[width=1\columnwidth]{figures/boi-position.pdf}
    \end{center}
\vspace*{-5mm}
    \caption{Fork of two chains. Solid lines connect blocks of $\pis$,
    and dashed lines connect blocks of $\pic$. In this configuration,
    blocks in dashed circles are aimless blocks of interest, and the block
    in the solid circle is a meaningful block of interest. Blocks B, C and E are
    aimless because they exist in $\pic$. Block A is aimless because it
    belongs to the subchain $\pis\{{:}\lca\}$}
    \label{fig:boi-position}
\vspace*{-4mm}
\end{figure}

In Figure~\ref{fig:boi-position} we display a fork of two proofs. Solid lines
connect blocks of $\pis$ and dashed lines connect blocks of $\pic$. By
examining which scenarios are meaningful based on different positions of the
block of interest, we observe that blocks \texttt{B}, \texttt{C} and \texttt{E}
do not qualify, because they are included in $\pic$. Block \texttt{A} is
included in $\pis\{{:}\lca\}$, which means that $\pic$ is an aimless contest
because the $\lca$ comes after the block of interest. Therefore, A is an
aimless block as a component of an aimless contest. Given this configuration,
the only meaningful block of interest is \texttt{D} and its predecessors (which
we leave out from this figure).

\noindent \textbf{Minimal forks.} By combining the above observations, we
derive that $\pic$ can be truncated into $\pic\{\lca{:}\}$ without affecting
the correctness of the protocol. We term this truncated proof $\pitr$.
Security is preserved by requiring $\pitr$ to be a \emph{minimal fork} of
$\pis$. A minimal fork is a fork chain that shares exactly one common block
with the main chain. A proof $\tilde\pi$, which is minimal fork of $\pi$, has
the following attributes:

\begin{enumerate}
\item $\pi\{lca\} = \tilde\pi[0]$
\item $\pi\{lca{:}\} \cap \tilde\pi[1{:}] = \emptyset$
\end{enumerate}

By requiring $\pitr$ to be a minimal fork of $\pis$, we prevent adversaries
from dispatching an augmented $\pitr$ to claim better score against $\pis$.
Such an attempt is displayed in Figure~\ref{fig:adversary-minimal-fork}.

\begin{figure}[h]
    \begin{center}
        \includegraphics[width=0.85\columnwidth]{figures/adversary-minimal-fork.pdf}
    \end{center}
\vspace*{-5mm}
    \caption{An adversary attests to contest with a malformed proof. Adversary
        proof consists of blocks \{A, X, B, C, D'\} that achieve better score
        against submit proof \{A, B, C, D\}. This attempt is rejected due to
        the minimal-fork requirement.}
    \label{fig:adversary-minimal-fork}
\vspace*{-5mm}
\end{figure}

In Algorithm~\ref{alg:minimal-fork}, we show how the minimal fork technique is
incorporated into our client replacing DAG and ancestors. In
Figure~\ref{fig:minimal-fork} we show how the performance of the client
improves. We use the same test case as in \emph{hash-and-resubmit}.

By applying the minimal-fork technique, he achieve a 55\% decrease in gas
consumption. \emph{Submit} phase now costs {4{,}700{,}000} gas, and
the \emph{contest} phase costs {4{,}900{,}000} million gas. This is a notable
result, since each phase now fits inside an Ethereum block.

\renewcommand{\genesis}{\textsf{G}}

\begin{algorithm}


    \caption{\label{alg:minimal-fork}The \textsf{NIPoPoW} client using the minimal fork technique}

    \begin{algorithmic}[1]

    \Contract{crosschain}
    \State $\textsf{events} \gets \bot;$ $\genesis \gets \bot$
    \Function{\sf initialize}{$\genesis_{remote}$}
        \State \genesis $\gets \genesis_{remote}$
    \EndFunction
    \Function{\sf submit}{$\pis$, $e$}
        \State \textsf{require}($\pis$[0] = $\genesis$)
        \State \textsf{require}($\textsf{events$[e]$} = \bot$)
        \State \textsf{require}($\textsf{valid-interlink}(\pis)$)
        \State \textsf{events$[e]$.hash} $\gets$ \textsf{H}($\pis$)
        \State \textsf{events$[e]$.pred} $\gets$
        \textsf{evaluate-predicate}(\textsf{$\pis$}, $e$)
    \EndFunction
    \Function{\sf contest}{$\pisa$, $\pitr$, $e$, $f$}
        \Comment{$f$: fork index}
        \State \textsf{require}($\pitr$[0] = $\pisa[f]$)
        \Comment{check fork head}
        \State \textsf{require}(\textsf{events}$[e]$ $\ne$ $\bot$)
        \State \textsf{require}(\textsf{events$[e]$.hash} $=$ \textsf{H}($\pisa$))
        \State \textsf{require}(\textsf{valid-interlink}($\pitr$))
        \State \textsf{require}(\textsf{minimal-fork}($\pisa$,
        $\pitr$, $f$))
        \State \textsf{require}(\textsf{score}($\pitr$)
            $>$ \textsf{score}($\pisa[f:]$))
        \State \textsf{events$[e]$.pred} $\gets$
            \textsf{evaluate-predicate}($\pitr$, $e$)
    \EndFunction
    \Function{\sf minimal-fork}{$\pi_1$, $\pi_2$, $f$}
        \For{$p\ in\ \pi_1$}
            \If{$p \in \pi_2$}
                \State\Return false
            \EndIf
        \EndFor
    \EndFunction
    \EndContract
    \vskip8pt
    \end{algorithmic}
\end{algorithm}



\begin{figure}[h]
    \begin{center}
        \includegraphics[width=1\columnwidth]{figures/minimal-fork.pdf}
    \end{center}
\vspace*{-5mm}
    \caption{Performance improvement using minimal fork (lower is better). The
        gas consumption is decreased by approximately 55\%.}
    \label{fig:minimal-fork}
\vspace*{-5mm}
\end{figure}
