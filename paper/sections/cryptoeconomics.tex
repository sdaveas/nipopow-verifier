\section{Cryptoeconomics}

We now present our economical analysis on RePoPoW. We have already discussed
that the NIPoPoW protocol is performed in distinct phases. In each phase,
different entities are prompted to act. As in SPV, we make the security
assumption that at least one honest node monitors the transaction to the
verifier contract in order to serve honest proofs. However, the process of
contesting a submitted proof by a honest node does not come without expenses.
Such an expense is the computational power a node has to consume in order to
fetch a submitted proof from the calldata and construct a contesting proof,
but most importantly, the gas that has to be paid in order to dispatch a
contesting proof to the Ethereum blockchain. Therefore, it is essential to
provide motives to nodes. On the contrary, adversaries have to be dishearten
from submitting invalid proofs.  We refer to the principle of promoting honest
actions against adversarial actions as "fairness".

In NIPoPoWs, fairness is addressed by the establishment of a monetary value
termed collateral. In \emph{submit} phase, the user pays this collateral in
addition to the expenses of the function call, and, if the proof is contested
successfully, the collateral is paid to the user that achieves to invalidate
the proof. If the proof is not contested, then the collateral is returned to
the original issuer. This treatment incentivizes nodes to participate to the
protocol, and discourages adversaries from joining. It is critical that the
collateral covers all the expenses of the entity issuing the contest.

The determination of a fair collateral is not trivial. Thorough analysis has to
be made regarding to the gas consumption of all involved phases, as well as the
immediacy of required actions. In Solidity, the priority of transactions is
determined by the gas price(ref) a user assigns to the underlying
transaction. This means that the user can chose the estimated time in which
transactions are published. Since the duration of contest period in NIPoPoWs is
bounded by a finite number of rounds $n$, the probability that the contesting
proof is included within the following $n$ blocks must be decisive. Otherwise,
it is possible for an invalid proof to be established due to the lack of
challenge. Albeit the user that initiates the submission may demand a direct
interaction, and thus selects a high gas price, this does not affect the value
of the collateral, as the burden of the node is not affected by the quickness
of initial proof's publication.

We examine various cases in which an invalid submission is followed by a
successful contest. Generally, we expect for an adversary to provide a proof of
a chain that is a fork of the honest chain at some point relatively close to
the tip. This is due to the fact that the ability of an adversary to sustain a
fork chain is exponentially weakened as the honest chain progresses. We
consider a fork of 100 blocks sufficient to describe an attempt of a power
adversary. However, we examine further cases. Our experiments include fraud
proofs of chains that fork a Bitcoin-like honest chain $100$, $100{,}000$ and
$650{,}000$ blocks prior to the tip. The last experiment essentially represents
the case of selfish mining(ref) from Bitcoin's genesis. We define $\textsf{Z}$
as the profit the node gains in case of a successful contest so that $
\text{collateral} = \textsf{Z} + \text{gas expended}$. We consider \textsf{Z} =
$0.1$ Ether (\$2) a sufficient amount. Different gas prices formulate different
costs for contest.

We used ETH Gas Station~\cite{eth-gas-station} to calculate the probabilities
of transactions inclusions with respect to gas price. In
Figure~\ref{fig:cryptoeconomics} we illustrate our economical analysis of out
client. Green and blue solid lines in each graph display the transaction cost
in USD for each phase as a function of the gas price. The solid red line in
each graph represents the collateral as a function of the probability of the
inclusion of contest transaction in the following 200 blocks. As this
probability approaches 1, the gas price needs to be increased. Dashed lines
illustrate the total cost of a submission in USD for several selections of
collateral depending on the desired value of gas price.  The selection of gas
price of the submission does not affect the collateral, but dictates the
immediacy of the transaction.

In Figure~\ref{fig:cryptoeconomics-100} we observe that the transaction cost of
a submission for a chain that equals the length of Bitcoin as a mid priority
transaction (80\% probability of getting accepted in the next 200 blocks) is
\$9.25, while the contest transaction in very high priority (100\% probability
of being included in the next 200 blocks) is \$7.99. The collateral in this
case is \$9.99 (contest cost+\textsf{Z}).

\begin{figure}[!h]
\begin{subfigure}{0.97\linewidth}
    \begin{center}
        \includegraphics[width=1\columnwidth]{figures/cryptoeconomics-100.pdf}
    \end{center}
    \caption{Collateral analysis for fork proof at index 100.}
    \label{fig:cryptoeconomics-100}
\end{subfigure}

\begin{subfigure}{0.97\linewidth}
    \begin{center}
        \includegraphics[width=1\columnwidth]{figures/cryptoeconomics-100K.pdf}
    \end{center}
    \caption{Collateral analysis for fork proof at index 100.000.}
    \label{fig:cryptoeconomics-100K}
\end{subfigure}

\begin{subfigure}{0.97\linewidth}
    \begin{center}
        \includegraphics[width=1\columnwidth]{figures/cryptoeconomics-650K.pdf}
    \end{center}
    \caption{Collateral analysis for fork proof at index 650.000.}
    \label{fig:cryptoeconomics-650K}
\end{subfigure}
\caption{Economical analysis of RePoPoW.}
\label{fig:cryptoeconomics}
\end{figure}
