\section{The Hash-and-Resubmit Pattern}

We now introduce a novel design pattern for Solidity smart contracts that
results into massive gas optimization due to the elimination of expensive
storage operations.

\textbf{Motivation.}
% This part is maybe too shallow. Consider deleting it. >>>> In the Ethereum
% blockchain, Turing-complete smart contracts were introduced. In order to
% prevent accidental or adversarial DoS phenomena such as infinite loops of
% code, contract invocations are bounded by an amount of gas units~\cite{wood,
% buterin}.  <<<<
It is essential for smart contracts to store data in the blockchain. However,
interacting with the storage of a contract is among the most expensive
operations of the EVM~\cite{wood, buterin}. Therefore, only necessary data
should be stored and redundancy should be avoided when possible. This is
contrary to conventional software architecture, where storage is considered
cheap. Usually, the performance of data access in traditional systems is
related with time. In Ethereum, however, performance is related to gas
consumption. Access to persistent data costs a substantial amount of gas, which
has a direct monetary value. One way to mitigate gas cost of reading variables
from the blockchain is to declare them public.  This leads to the creation of a
\emph{getter} function in the background, allowing free access to the value of
the variable. But this treatment does not prevent the initial population of
storage data, which is significantly expensive for large size of data.
Towards the goal of implementing gas-efficient smart contracts, several
patterns have been proposed~\cite{contract-opt-1, contract-opt-2,
contract-opt-3}.

By using the \emph{hash-and-resubmit} pattern, large storage structures are
omitted entirety, and are contained in memory which results into vastly
improved performance. When a function call is performed, the arguments and
signature of the function is included in the transactions field of the body of
a block. The contents of blocks are public to the network, therefore this
information is locally available to full nodes. By simply observing blocks, a
node retrieves data sent by other users. To interact publicly with the data
sent by other users without the utilization of on-chain storage, the node
\emph{resends} the observed data to the blockchain. Consequently, this data
does not need to be maintained in storage. The concept of resending data would
be redundant in conventional systems. However, it is very efficient to
utilizing the technique in Solidity because it leverages operations in memory
rather than storage, which are notable cheaper.
\begin{figure*}[h]
    \begin{center} \includegraphics[width=1\textwidth]{figures/har-pattern.pdf}
    \end{center}

    \caption{The \emph{hash-and-resubmit} pattern. In stage 1, an invoker calls
        \proc$_1$($\data_0$). $\data_0$ is processed on-chain and $\data$ is
        generated. The signature of $\data$ is stored in the blockchain as the
        digest of a hash function \textsf{H}(). In stage 2, a full node that
        observes invocations of $\proc_1$ retrieves $\data_0$, and generates
        $\data$ by performing the analogous processing on $\data_0$
        \emph{off-chain}. An adversarial observer potentially alters $\data$.
        Finally, in stage 3, the observer invokes $\proc_2$($\data^*$). In
        $\proc_2$, the validation of $\data$ is performed, reverting the
        function call if the signatures of originally submitted $\data$ does
        not match the signature of $\data^*$. By applying the
        \emph{hash-and-resubmit pattern}, only fixed-size signatures of data
        need to be maintained on the blockchain replacing arbitrarily large
        structures.}

        \label{fig:har-pattern}
\end{figure*}

\noindent
\textbf{Applicability.}
We now list the cases in which the \emph{hash-and-resubmit} pattern is
efficient to use:
\begin{enumerate}
    \item To reduce gas inefficiencies caused by extensive read/write storage
        operations and to make smart contracts that exceed block gas limit
        practical.
    \item To interact with smart contract depending on prior actions of other
        users.
    \item To leverage off-chain operations.
\end{enumerate}

\noindent \textbf{Participants and collaborators.} The first participant is the
smart contract $\contract$ that accepts function calls. Another participant is
the invoker $\invoker$, who dispatches arbitrary data $\data_0$ to $\contract$
via a function \texttt{\proc$_1$}($\data_0$). Note that $\data_0$ are
potentially processed in $\proc_1$, resulting to $\data$. The last participant
is the observer $\observer$, who is a full node that observes transactions
towards $\contract$ in the blockchain. This possible because nodes maintain
the blockchain locally. After observation, $\observer$ retrieves data $\data$.
Finally, $\observer$ acts as an invoker by making a new interaction with
$\contract$, \texttt{\proc$_2$}($\data$). Since this is an off-chain operation,
a malicious $\observer$ can alter $\data$ before interacting with $\contract$.
We will denote the potentially modified $\data$ as $\datas$. The verification
that $\data = \datas$, which is a prerequisite for the secure functionality of
the underlying contract consists a part of the pattern and is performed in
\texttt{\proc$_2$}($\datas$).

\noindent \textbf{Implementation.} The implementation of this pattern is
divided in two parts. The first part covers how $\datas$ is retrieved by
$\observer$, whereas in the second part the verification of $\data=\datas$ is
realized. The challenge here is twofold:

\begin{enumerate}

    \item Availability: $\observer$ must be able to retrieve $\data$ without
        the need of accessing on-chain data.

    \item Consistency: $\observer$ must be prevented from dispatching $\datas$
        that differs from the originally submitted $\data$.

\end{enumerate}

\noindent
\emph{Hash-and-resubmit} technique is performed in two
stages to face these challenges: (a) the \emph{hash} phase, which addresses
\emph{consistency}, and (b) the \emph{resubmit} phase which addresses
\emph{availability} and \emph{consistency}.

\noindent \textsf{Addressing availability:} During \emph{hash} phase,
$\invoker$ makes the function call \texttt{\proc}$_1$($\data_0$). This
transaction, which includes a function signature (\texttt{\proc$_1$}) and the
corresponding data ($\data_0$), is added in a block by a miner. Due to
blockchain's transparency, the observer of \texttt{\proc}$_1$, $\observer$,
retrieves a copy of $\data_0$, without the need of accessing contract data. In
turn, $\observer$ performs \emph{locally} the same set of on-chain instructions
operated on $\data_0$ generating $\data$. Thus, availability is addressed
through observability.

\noindent \textsf{Addressing reliability:} We prevent an adversary $\observer$
from altering $\datas$ by storing the \emph{signature} of $\data$ in contract's
state during the execution of \texttt{\proc$_1$($\data$)} by $\invoker$. In the
context of Solidity, a signature of a structure is the digest of the
structures's \emph{hash}. The pre-compiled \texttt{sha256} is convenient to use
in Solidity, however we can make use of any cryptographic hash function
\textsf{H()}: \[\textsf{hash} \gets \textsf{H}(\textsf{d})\] Then, in
\emph{rehash} phase, the verification is performed by comparing the stored
digest of $\data$ with the digest of $\datas$.
\[\textsf{require}(\textsf{hash} = \texttt{H}(\datas))\] \noindent In Solidity,
the size of digests is 32 bytes. To persist such a small value in contract's
memory only adds a constant, negligible cost overhead.

We illustrate the application of the \emph{hash-and-resubmit} pattern in
Figure~\ref{fig:har-pattern}.

\noindent \textbf{Sample.} We now demonstrate the usage of the
hash-and-resubmit pattern with a practical example. We create a smart contract
that orchestrates a game between two players, $\pla$ and $\plb$. The winner is
the player with the most valuable array. The interaction between players
through the smart contract is realized in two phases: (a) Submit phase and (b)
Contest phase.

\noindent \textsf{Submit phase:} $\pla$ submits an N-sized array, $\arra$ and
becomes the $\holder$ of the contract.

\noindent \textsf{Contest phase:} $\plb$ submits $\arrb$. If $\arrb$ $>$
$\arra$, then the $\holder$ of the contract is changed to $\plb$. We provide a
simple implementation for the $>$ operator between arrays, but we can consider
any notion of comparison between arrays, since the pattern is abstracted from
such implementation details.

We make use of the \emph{hash-and-resubmit} pattern by prompting $\plb$ to
provide \emph{two} arrays to the contract during contest phase: (a) $\arras$,
which is the originally submitted array by $\pla$, possibly modified by $\plb$,
and (b) $\arrb$, which is the contesting array.

We provide two implementations of the above described game.
Algorithm~\ref{alg.compare-storage} is the storage implementation and
Algorithm~\ref{alg.compare-memory} is the implementation embedding the
\emph{hash-and-resubmit} pattern.

\begin{algorithm}
    \label{alg:compare-storage}
    \caption{\textsf{best array} using storage}
    \begin{algorithmic}[1]

    \Contract{best-array}
        \Function{\sf initialize}{}
            \State{$\textsf{best} \gets \emptyset$;
                   $\textsf{holder} \gets \emptyset$}
        \EndFunction
        \Function{\sf submit}{$a$}
        \State \textsf{best} $\gets a$
            \Comment{array saved in storage}
            \State \textsf{holder $\gets$ msg.sender}
        \EndFunction

        \Function{\sf contest}{$a$}
            \State \textsf{require}(\textsf{compare}($a$))
            \State \textsf{holder} $\gets$ \textsf{msg.sender}
        \EndFunction

        \Function{\sf compare}{$a$}
            \State \textsf{require}($|a|$ $\geq$ $|$\textsf{best}$|$)
            \For{$i$ : $|$\textsf{best}$|$}
            \If{$a[i]$ $\leq$ \textsf{best}[i]} \Return{false}
                \EndIf
            \EndFor
            \State \Return{true}
        \EndFunction
        \EndContract
        \vskip8pt
    \end{algorithmic}
\end{algorithm}

\begin{algorithm}
    \label{alg:compare-memory}
    \caption{\textsf{best array} using hash-and-resubmit pattern}
    \begin{algorithmic}[1]
        \Contract{best-array}
        \Function{\sf initialize}{}
        \State{$\textsf{hash} \gets \emptyset$;
               $\textsf{holder} \gets \emptyset$}
        \EndFunction
        \Function{\sf submit}{$\arra$}
        \State $\textsf{hash} \gets \textsf{H}(\arra)$
            \Comment{hash saved in storage}
            \State \textsf{holder} $\gets$ \textsf{msg.sender}
        \EndFunction

    \Function{\sf contest}{$\arra^*$, $\arrb$}
    \State \textsf{require}(\textsf{hash256}($\arra^*$) $=$ $hash$)
        \Comment{validate $\arra^*$}
        \State \textsf{require}(\textsf{compare}($\arra^*$, $\arrb$))
        \State \textsf{holder} $\gets$ \textsf{msg.sender}
    \EndFunction
    \Function{\sf compare}{$\arra^*$, $\arrb$}
        \State \textsf{require}($|\arra^*|$ $\geq$ $|\arrb|$)
        \For{$i : |\arra^*|$}
        \If{$\arra^*[i] \leq \arrb[i]$} \Return{false}
            \EndIf
        \EndFor
    \EndFunction
    \State \Return{true}
    \EndContract
    \vskip8pt
    \end{algorithmic}
\end{algorithm}


\noindent \textbf{Gas analysis.} The gas consumption of the two implementations
is displayed in Figure~\ref{fig:har-example}. By using the
\emph{hash-and-resubmit} pattern, the overall gas consumption for
\textsf{submit} and \textsf{contest} is decreased by 95\%. This significantly
affects the efficiency and applicability of the contract. Note that, the
storage implementation exceeds the Ethereum block gas limit\footnote{As of July 2020, the Ethereum block gas limit approximates 10,000,000 gas units} for
arrays of size 500 and above, contrary to the optimized version, which
consumes approximately only $1/10^{th}$ of the block gas limit for arrays of
1000 elements.

\begin{figure}[h!]
\begin{center}
\includegraphics[width=1 \columnwidth]{figures/har-example.pdf}
\end{center}
\caption{Gas-cost reduction using the \emph{hash-and-resubmit} pattern. By
    avoiding gas-heavy storage operations, the aggregated cost of
    \textsf{submit} and \textsf{contest} is decreased significantly by 95\%.}
\label{fig:har-example}
\end{figure}


\noindent \textbf{Variations.} Now consider a variation of the above game, in
which $\pla$ calls \texttt{\proc$_1$(}$\arra$\texttt{)}, and then calls
\texttt{pickSpan(}$m, n$\texttt{)} that determines the span of $\arra$ which
can be contested. In reality, $\plb$ only needs to re-send $\arras[m:n]$ in
order to perform the comparison $\arra[m:n] < \arrb$. However, the digest of
$\arra$ is calculated by hashing the entire structure. Therefore, the
$resubmit$ phase cannot be successfully performed by rehashing $\arras[m:n]$,
because \texttt{H(}$\arra$\texttt{)} $\ne$ \texttt{H(}$\arra[m:n]$\texttt{)}.

An intuitive approach to address such scenarios in order to facilitate
selective dispatch of structure segments is to adopt different hashing schemas
that utilize constructions such as Merkle Trees (ref) or Merkle Mountain Ranges
(ref). We will proceed by only mentioning Merkle Trees, but since the
respective operations for proving and validating roots is of the same
complexity, the same principle applies for Merkle Mountain Ranges.

In the Merkle variation of the pattern, which we term
\emph{merkle-hash-and-resubmit}, the signature of $\arra$ is generated by
constructing the Merkle Tree Root (MTR) of $\arra$ in contract's state. In
\emph{resubmit} phase, $\arra[m:n]$ is dispatched, accompanied by the siblings
that reconstruct the MTR of $\arra$ in order to perform the variation of
$\arras[m:n]$.

\begin{figure*}[h]
    \begin{center}
        \includegraphics[width=0.8\textwidth]{figures/merkle-har.pdf}
    \end{center}
    \caption{\textbf{I.} The calculation of root in \emph{hash} phase.
    \textbf{II.} The verification of the root in \emph{resubmit} phase.
    \textsf{H}($k$) denotes the digest of element $k$. \textsf{H}($kl$) denotes the
    result of \textsf{H}(\textsf{H}($k$) $|$ \textsf{H}($l$))
}
    \label{fig:merkle-har}
\end{figure*}

This variation of the pattern removes the burden of sending redundant data,
however it implies on-chain construction and validation of the Merkle
construction. In order to construct a MTR for an array \textsf{a} of size
\textsf{n}, the hash function \textsf{H}($|$\textsf{a}$[n]|$) needs to be
called approximately $2^{\lceil log_2(n) + 1 \rceil}$ times. In order to
reconstruct the Merkle Tree Root for the verification, approximately $\lceil
log_2{n} \rceil$ calls to \textsf{H}(.) are needed. The process of constructing
and verifying the MTRoot is displayed in Figure ~\ref{fig:merkle-har}. Note
that both operations need to be performed on-chain.

In Solidity, hashing operations are expensive compared to memory cost. An
invocation of \textsf{sha256}(\textsf{a}), copies the data in memory, and then
a \textsf{CALL} instruction is performed by the EVM to the pre-compiled
\textsf{sha256} contract. In the current state of the EVM, \textsf{CALL} costs
700 gas units, and the gas paid for every word when expanding memory is 3 gas
units~\cite{wood}. Costs of all related operations are listed in
Table~\ref{tab:operations-gas}. Due to this large discrepancy in gas costs, it
is extremely more efficient to send data of size $|$\textsf{a}$|$ and perform
$1 \times \textsf{sha256}(a)$ than sending data of size $log_2(|\textsf{a}|)$
and perform $2^{log_2(\textsf)} \times$\textsf{sha}(1) operations. A different
cost policy applies for \textsf{keccak}~\cite{keccak} hash function, where
hashing costs 30 gas units plus 6 additional gas far each word for input data.
Although the usage of \textsf{keccak} dramatically increases the performance in
comparison with \textsf{sha256}, plain rehashing generally performs better.

\begin{table}[]
\begin{tabular}{|c|c|}
\hline
\textbf{Operation} & \textbf{Gas cost} \\ \hline
\textsf{load}($\data$)            & $68 \times |\data|$          \\ \hline
\textsf{mem}($\data$)             & $4 \times |\data|$           \\ \hline
\textsf{sha256}($\data$)          & $4 \times |\data| + 700$     \\ \hline
\textsf{keccak}($\data$)          & $6 \times |\data| + 30$      \\ \hline
\end{tabular}
\caption{Gas cost for operations as of July 2020.}
\label{tab:operations-gas}
\end{table}


In Table~\ref{tab:har-vs-mhar} we display the operations needed for hashing and
verifying the underlying data for both variations of the pattern as a function
of data size. In Figure~\ref{fig:har-vs-mhar} we demonstrate the gas
consumption for $\data_0$ = 10KB and $\data$ varing from $0.01
\times|\data_0|$ to $2\times|\data_0|$.

\newcommand{\mydata}{\data}

\begin{table}
\centering
\begin{tabular}{|c|c|c|}
\hline
\textbf{\begin{tabular}[c]{@{}c@{}}phase per\\variance\end{tabular}} &
\textbf{\begin{tabular}[c]{@{}c@{}}plain hash\\and resubmit\end{tabular}} &
\textbf{\begin{tabular}[c]{@{}c@{}}merkle hash\\ and resubmit\end{tabular}} \\ \hline
\textbf{hash} &
\textsf{H}($\mydata$) &
\begin{tabular}[c]{@{}c@{}}
    \textsf{H}($\mydata_{elem}$) $\times\ |\mydata|$ \\ \textsf{H}(digest)
$\times\ (|\mydata|-1)$

\end{tabular} \\ \hline
\textbf{resubmit} &
\textsf{load}($\mydata$) + \textsf{H}($\mydata$) &
\begin{tabular}[c]{@{}c@{}}
    \textsf{load}($\mydata[m{:}n])$ + \\
    \textsf{load}($siblings$) + \\
    \textsf{H}($\mydata[m{:}n])$ + \\
    \textsf{H}($digest$)$\times |siblings|$
\end{tabular} \\ \hline
\end{tabular}

\caption{Summary of operations for \emph{hash-and-resubmit} pattern variations.
$\mydata$ is the product of on-chain operations and $\mydata_{elem}$ is an
element of $\mydata$. \textsf{H} is a hash function, such as \textsf{sha256}
or \textsf{keccak}, $digest$ is the product of \textsf{H}(.) and $siblings$ are
the siblings of the Merkle Tree constructed for $\mydata$.
}

\label{tab:har-vs-mhar}
\end{table}


\begin{figure}[h]
    \begin{center}
        \includegraphics[width=1\columnwidth]{figures/har-vs-mhar.pdf}
    \end{center}
    \caption{Trade-offs between \emph{hash-and-resubmit} variations. In the
    vertical axis the gas consumption is displayed, and in vertical axis the
    size of $\data$ as a function of $\data_0$. The size of $d_0$ is 10KB
    bytes, and the hash function we used is pre-compiled \texttt{sha256}.}
    \label{fig:har-vs-mhar}
\end{figure}

\noindent \textbf{Consequences.} The most obvious consequence of applying the
\emph{hash-and-resubmit} pattern variations is the circumvention of storage
structures, a benefit that saves a substantial amount of gas, especially in the
cases where these structures are large. To that extend, smart contracts that
exceed the Ethereum block gas limit become practical. Furthermore, the pattern
enables off-chain transactions, significantly improving the performance of
smart contracts.

\noindent \textbf{Known uses.} To our knowledge, we are the first to combine
the notion of the transparency of the blockchain with data structures
signatures to eliminate storage variables from Solidity smart contracts by
resubmitting data.

\noindent \textbf{Enabling NIPoPoWs.} We now present how the
\emph{hash-and-resubmit} pattern can used in the context of the NIPoPoW
superlight client. Similar to the aforementioned example, the NIPoPoW verifier
adheres to a submit-and-contest-phase schema, and the inputs of the functions
are arrays that are processed on-chain.

In \emph{submit} phase, a \emph{proof} is submitted, which can be contested by
another user in \emph{contest} phase. The user that initiates the contest,
monitors the traffic of the smart contract~\cite{nipopows}. This is a logical
assumption as mentioned in the NIPoPoW paper. The input of \textsf{submit}
function includes the submit proof ($\pis$) that indicates the occurrence of an
\emph{event} ($e$) in the source chain, and the input of \textsf{contest}
function includes the contesting proof ($\pic$). A successful contest of $\pis$
is realized when $\pic$ has a better score. The score evaluation process is
irrelevant to the pattern and remains unchained. The size of proofs is dictated
by the value $m$. We consider $m$ = 15 sufficiently secure.

In previous work~\cite{gglou}, NIPoPoW proofs are maintained on-chain,
resulting to extensive storage operations that limit the applicability of the
contract considerably. In Algorithm~\ref{alg:har-nipopow} we show how
hash-and-resubmit pattern is embedded into the NIPoPoW client. In
Figure~\ref{fig:har-nipopow}, we display how results of the
\emph{hash-and-resubmit} implementation differentiate from previous work for
the aggregated cost of \emph{submit} and \emph{contest} phases.  We observe
that by using the \emph{hash-and-resubmit} pattern, we achieve to increase the
performance of the contract 40\%. This is a decisive step towards creating a
practical superlight client.

\begin{algorithm}

    \caption{\label{alg:har-nipopow}The \textsf{NIPoPoW} client using hash-and-resubmit pattern}
    \begin{algorithmic}[1]

    \Contract{crosschain}
    \State $\textsf{events} \gets \bot;$ $\genesis \gets \bot$
    \Function{\sf initialize}{$\genesis_{remote}$}
        \State $\genesis$ $\gets \genesis_{remote}$
    \EndFunction
    \Function{\sf submit}{$\pis$, $e$}
        \State \textsf{require}($\pis$[0] = $\genesis$)
        \State \textsf{require}($\textsf{events$[e]$} = \bot$)
        \State \textsf{require}($\textsf{valid-interlinks}(\pi)$)
        \State \textsf{DAG} $\gets$ \textsf{DAG} $\cup$ $\pis$
        \State \textsf{events$[e]$.hash} $\gets$ \textsf{H}($\pis$)
        \Comment{enable pattern}
        \State \textsf{ancestors} $\gets$ \textsf{find-ancestors()}
        \State \textsf{events$[e]$.pred} $\gets$
            \textsf{evaluate-predicate}(\textsf{ancestors}, e)
        \State \textsf{ancestors} $=$ $\bot$
    \EndFunction
    \Function{\sf contest}{$\pisa$, $\pic$, $e$}
        \Comment{provide proofs}
        \State \textsf{require}(\textsf{events$[e]$.hash} $=$ \textsf{H}($\pisa$))
        \Comment{verify $\pisa$}
        \State \textsf{require}($\pic$[0] = $\genesis$)
        \State \textsf{require}(\textsf{events}$[e]$ $\ne$ $\bot$)
        \State \textsf{require}(\textsf{valid-interlinks}($\pi_{cont}$))
        \State $lca$ = \textsf{find-lca}($\pisa$, $\pic$)
        \State \textsf{require}(\textsf{score}($\pic[{:}lca]$)
        $>$ \textsf{score}($\pisa[{:}lca]$))
        \State \textsf{DAG} $\gets$ \textsf{DAG} $\cup$ $\pic$
        \State \textsf{ancestors} $\gets$ \textsf{find-ancestors}(\textsf{DAG})
        \State \textsf{events$[e]$.pred} $\gets$
            \textsf{evaluate-predicate}(\textsf{ancestors}, $e$)
        \State \textsf{ancestors} $=$ $\bot$
    \EndFunction
    \EndContract
    \vskip8pt
    \end{algorithmic}
\end{algorithm}



\begin{figure}[!h]
    \begin{center}
        \includegraphics[width=1\columnwidth]{figures/har-nipopows.pdf}
    \end{center}
    \caption{Performance improvement using hash-and-resubmit pattern in
    NIPoPoWs related to previous work for a secure value of $m$. The gas
    consumption decreased by approximately 40\%.}
    \label{fig:har-nipopow}
\end{figure}
